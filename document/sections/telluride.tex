
\subsection{PushBots Racing Competition: Visually Guided Reactive Navigation}


\subsection{Robotic \textquotedbl{}sheep\textquotedbl{} effect: One pushbot follows
the other}

Navigation SLAM and Planning? 


\subsection{Make sure we do not run into things}


\subsection{Seek and destroy }

Hawk Dove Model 

Trex


\subsection{interface for controlling the EMB14 tank and EMB}


\subsubsection{Introduction EMB}

While recent studies have made strides in biologically grounded technical
models of individual mind and body components {[}CN{]}, there has
been relatively less progress made on a system level. What progress
has been made tends to focus on either high-level biological descriptions
or on low-level technical implementation. For the merger of these
disparate goals, robotics is generally accepted as a suitable medium.
Robotics implementations, however, tend toward architectures that
are goal oriented rather than trying to mimic how tasks are accomplished
in biological systems. Success is measured by the competence of the
robot. Since the science of the neurobiology of behavior lags our
technologies, it is easy to take shortcuts to achieve criterion goals.
Building robots that mimic biological models such as the Distributed
Adaptive Control (DAC) architecture using tools that are optimized
for neural simulation such as the Nengo programming system from the
Neuroengineering Framework (NEF) informs the DAC theory and allows
iterative improvement in both theory and robot competence. In particular
we would like to test a model that combines, low level genetically
determined behavioral response biases with adaptive classical and
operational conditioning that are based upon known facts about cerebellar
and hippocampal architecture and function. Furthermore, we wish to
model neurally inspired prefrontal short term and long term memory
acquisition and the generation of plans from memory that can augment
or override conditioned responses. In effect we wish to simulate all
the key neuro-cognitive elements of a complete adaptive organism with
neural models. An airplane emulates and surpasses aspects of a bird\textquoteright{}s
flight, but leaves out some of the most fundamental adaptive capabilities
of a bird. We seek to integrate high fidelity modeling even if it
is at the expense of some sacrifice in criterion performance in the
initial instantiations. An adjunct goal is of course to train and
inform a new generation of neuromorphic researchers in the DAC model
and in the use of the tools available from NEF. This will eventually
allow these researchers to combine brain and behavioral models in
a modular way, and will thereby lead to an acceleration in the development
and deployment of advanced models and robots much as scientific computing
and statistical libraries have contributed to many fields of study.

-------------------------------------

-\textquotedblright{}To take our understanding of individual body/brain
components and further our understanding on a systems level\textquotedblright{}
{[}emb website{]}

-\textquotedblright{}This workshop will address the challenge of understanding
and building embodied neuromorphic real-world architectures of perception,
cognition and action through both presentations, discussion and concrete
experimentation.\textquotedblright{} {[}Emb web{]} Specifically we
intend to program an autonomous robot with a DAC (Distributed Adaptive
Control) architecture using a combination of IQR, Python and Nengo
(NeuroEngineering framework) tools, and have that robot learn to navigate
and forage in an artificial environment searching for faux reinforcement
while avoiding faux aversive stimuli or predators. {[}srd{]}


\subsubsection{Methods EMB}

The goal of this study is to create a biologically-grounded foraging
robot. The distributed adaptive control (DAC) {[}1{]} model will be
used as the systems level structure. This structure will be divided
among different functions defined in neurons through Nengo and the
corresponding Neural Engineering Framework (NEF) model {[}5,24{]}.
By using Nengo to rapidly prototype and implement modules of neural
algorithms while structuring these modules within the DAC hierarchy,
we hope to achieve a high degree of biological accuracy.

The Neural Engineering Framework The Neural Engineering Farmework
(NEF) is a means of implementing cognitive processes within a biological
substrate. The NEF enables the modeling of large scale networks of
individual neuron models. These large scale models have been used
to build models ranging from single cell activity {[}23{]} to age-related
cognitive decline {[}22{]}. These models that accurately explain and
predict real-world neural behavior. It is important to stress that
this ability was not an explicit constraint upon the original work.
Instead it is a consequence of the inherent biological constraint
of the model. These large-scale models are, generally speaking, created
with three guiding definitions.

The first of these definitions defines an ensemble of neurons which
are used to encode information as a population. A time-varying numeric
vector represents this ensemble of neurons. This ensemble is expected
to possess non-linear encoding due to the differences in tuning-curves
of the individual neurons. The output is then the filtered output
of the spiking of these neurons, summed with an uneven weighting to
create a linear output {[}5{]}.

Transformation, the second definition, defines how populations communicate
with other populations. If population A is communicating with population
B, the transformation matrix will begin with A\textquoteright{}s decoding
weights, end with B\textquoteright{}s encoding weights, and the middle
will be defined by any linear transformation. This linear transformation
term allows the modeling of arbitrary mathematical functions {[}5{]}.

Finally, dynamics is the term used to describe the recurrent connections
between populations. These recurrent connections allow the populations
to behave as state-variables in a dynamic system {[}5{]}.

Nengo is the software platform through which we implement the NEF
model and simulate these large populations of neurons. Nengo provides
not only an interactive means for modeling these many populations
of neurons, but it also provides a means for interfacing multiple
functional blocks.

Distributed Adaptive Control FIGURE T2 is a graphical representation
of the distributed adaptive control (DAC) model. This model represents
all layers of an adapative mammalian nervous system and provides a
systematic design methodology for creating and analyzing neural-controlled
robots with complex behaviors. In previous studies, these robotic
and mechatronic designs based upon DAC have lent insight to the workings
of actual biological systems and fine-tuning of the DAC theory. These
insights arise as a consequence of creating systems within the tight
biological grounding of the DAC model.

The DAC model consists of 4 levels with 3 columns each in a matrix.
The levels represent successively higher levels of cognition and learning.
The columns represent coupling from the bottom up and from the top
down within domains of sensing, state maintenance, and behavior.

More specifically, the leftmost blue column, or exosensing column,
represents influence from the physical world including basic perception
and perceptual abstraction and episodic memory in one column. The
center green endosensing column represents internal states and motivational
states based upon needs, values and their effect on decisions and
plans. The final column red column is for action, action gating, and
planning. Of key importance to the model is that it is embodied in
a system with sensors and effectors that is interacting with a world.

Spanning these columns are four rows categorizing the blocks\textquoteright{}
general functionality within the entity and representing capabilities
that build upon one another. The somatic layer deals with the self
as a physical body and the sensations it receives. In the context
of the robot, this layer is comprised of the various sensors which
are used by the robot to interact with the world around it. The reactive
layer controls the stereotypical behaviors or reflexes of the animal/animat
in response to species salient stimuli. These specific behaviors have
the property of being proportional to the stimuli received from the
somatic layer. The adaptive layer is what allows a system to learn
instead of being completely reactive. This Hebbian style learning
is enabled through the storage of former outcomes resulting from reactions
to specific stimuli. The contextual layer enables even more advanced
forms of learning through additional long-term memory units.

Layer based Implementation The layers of the DAC model were used to
provide the framework for the design of an autonomous foraging robot.
In this subsection, the implementation details will be explained in
the order that they were integrated into the system. We use IQR and
Nengo with Python to implement the model for the robot. IQR was used
for the Cerebellum and the Hippocampus models, and are long term development
projects {[}11{]}. The Reactive layer and the high level planning
were implemented using Nengo from NEF. Different parts of the overall
model were run on different PCs with different operating systems and
different software environments.

Allostatic Control: The allostatic controls represent the endosensing
function of the reactive layer within the DAC model and its influence
on behavior. It can be thought of as the organism's innate decision
making system to balance needs. If hungry, eat. If scared, run. If
both things occur, there must be some innate prioritization or some
higher level learned arbitration. Otherwise, there can be response
conflict that meets no needs. This control attempts to achieve an
internal allostatic equilibrium by optimal balance of all needs and
resources by way of reactive and stereotypical behavior{[}8{]}. The
job of these allostatic controls is to match the processes coupling
internal and external sensations of the robot which come from the
various sensors to the behavioral degrees of freedom available at
the somatic level. In our robot model these sensors include three
proximity and touch sensors mounted to the front of the tank robot
as well as a DVS camera {[}13{]}, compass, and velocity inferred from
information provided by the attached Pushbot.

The allostatic controls are not meant to provide a binary value (ie:
hungry/full, energetic/tired\ldots{}), but instead are meant to provide
an analog value which varies according to the intensity of the perceived
sensation. A relatable example of this is how persons react to the
tactile sensation of heat. If a person touches a car hood that has
been sitting in the sun, they will likely perceive some degree of
heat intensity. If the hood is not particularly hot, they may choose
to continue touching the hood (perhaps to open it and check an engine
component). Conversely, if the hood is hot enough to burn one\textquoteright{}s
hand, the person will likely withdraw their hand to avoid damaging
their skin. These opposing reactions are examples of stereotypical
behaviors. However, if there is some sort of urgent need that motivated
the original action, then the degree of this tactile sensation will
be utilized to determine if it is worth incurring the burn damage
or if a new plan is needed. This planning is performed at a higher
level within the DAC model, but it could not be done without the analog
value provided by the allostatic controls.

Vision GOAL OF FEATURE EXTRACTION

AER DESCRIPTION

Within this project, we are utilizing Address event Representation
(AER) with a eDVS {[}13{]} camera system. AER is a way of representing
neural spiking events and sending them over a shared bus. AER is also
often referred to as being an event-based protocol. AER conserves
bus bandwidth by only transmitting a code when an event happens rather
than scanning out the whole state of system activity repeatedly with,
for example, a raster scan. The code sent for an event is an address
that represents where the event originated from, eg, the address of
a spiking neuron. In the simplest implementations listeners on the
same AER bus must look for the addresses that are of significance
to them (their receptive field). Refinements of AER include adding
time stamps, hierarchical event routing, additions of parameters that
describe the event in more detail, and inclusion of geometry in the
address code, e.g. x,y location of the event source on a silicon retina,
x,y,z,,t location in a brain volume {[}14,15,16,17{]}. 

Events are created asynchronously in the AER framework as individual
neurons decide to spike. Thus, there is a variable demand for the
AER bus that often results in collisions. These are handled by a)
using a bus arbitration scheme that enforces single file access to
the bus, and b) using a bus that operates at high speeds to accommodate
many events in rapid succession, c) overlapped arbitration and event
transfer. In this manner, events are transmitted with little timing
jitter if the bus bandwidth is sufficient. In large networks of neurons
care must be taken to ensure that proper axonal delays are represented
where STDP (spike time dependent plasticity) is included {[}18{]},
or where temporal dynamics are an issue as in polychronous networks
{[}19{]}. This may require event queing, analog delays, hierarchical
routing, or other methods {[}20,21{]}. AER systems are only beginning
to test their limits of scalability to represent traffic along billions
of axons in systems with significant numbers of highly interconnected
neurons as researchers start to include STDP and dynamics related
time delays in their neuromorphic models.

BRIEF SUMMARY OF {[}7{]}

Adaptive:

Planning:

Goal:

Results

Discussion

Temporary Figure Storage

{[}Place holder image until original added back in{]}

Temp Fig 1: Shows a hardware-oriented view of our implementation,
and our integration schedule. We have multiple computers each doing
a different part of the DAC processing. So not only is the model distributed,
so also is the implementation for practical reasons. We are using
3D printed tank chassis driven by 4 motors with tank-like tracks.
On this we have mounted 3 subsystems. One is a motor driver board
that accepts commands over a local link from a legobot EV3 {[}25{]}
that has been stripped down to the bare essentials \textquoteleft{}brick\textquoteright{}
with CPU and communication. The lego bot passes motor commands from
remote sources to the motors and gathers proximity sensor status to
forward to the reactive layer. The reactive layer is represented by
another robot called the \textquoteleft{}pushbot\textquoteright{}
developed for the Institute for Neuroinformatics, ETH by INIlabs or
Zurich {[}26{]}. So not only is the model distributed, so also is
the implementation for practical reasons. We are using 3D printed
tank chassis driven by 4 motors with tank-like tracks. On this we
have mounted 3 subsystems. This is a true 'Rube Goldberg' implementation
because we had to improvise.

One subsystem on the tank is a motor driver board that accepts commands
over a local USB link from a legobot EV3 {[}25{]} that has been stripped
down to the bare essentials \textquoteleft{}brick\textquoteright{}
with A7 CPU and communication. The legobot passes motor commands originating
from a laptop simulating the reactive layer and gathers proximity
sensor status from the tank's on-board module to forward to the reactive
layer using UDP protocol over wifi. The reactive layer is supplied
with compass, DVS video sensor information, and battery voltage level
from another tank-mounted robot called the \textquoteleft{}pushbot\textquoteright{}
developed for the Institute for Neuroinformatics, ETH, and Univ. Zurich
by INIlabs{[}26 http://www.inilabs.com/company{]}. This also uses
UDP over wifi. These inputs are aggregated in the reactive layer which
processes battery voltage level as a measure of overall system readiness
and health, reacts to proximity issues, and looks for good and bad
stimuli in the environment indicated by flickering LEDs. It reactively
approaches and avoids these by sending motor commands to the leobot
to forward to the tank motor drivers.

The proximity information, the pushbot motor control decisions, the
video feed, and the LED location information is also forwarded to
the cerebellum simulation running on another laptop while the compass
heading and the averaged pushbot motor drive is sent to a hippocampus
simulation running on a third laptop. These too are wifi connections
using UDP protocol. The power level is also forwarded to another planning
task that keeps track of hippocampal output, power level, and goals
in order to output a direction command sent back to the reactive layer.
The cerebellum learns conditioned responses and sends to the reactive
simulation motor commands to add onto those of the reactive layer.
The cerebellar commands result from anticipating actions to perform
using signs in the video feed (classical conditioning). Meanwhile
the hippocampus simulation is learning a grid/place map and informing
the planning module so that it can chose a direction. The planner
direction is added to the cerebellum output that is added to the reactive
output and sent to the motor command relay station in the legobot.
So the motors are driven by a linear combination of input from reactive
(unconditioned), cerebellum (conditioned) and planner (cognitive).

Temp FIg 2: DAC diagram stolen from {[}1{]}. IT WOULD BE GOOD TO HAVE
THE ORIGINAL FILE. Alternatively, maybe we could make a simplified
version that suits the needs of this paper?

-General Neuromorphic Robotic Framework description -\textquotedblright{}One
reason for the lack of progress in understanding the interrelationship
of behaviour and perception is experimental intractability.... Our
approach was to bypass the animal experimental diffi{}culty by using
a mobile robot, for which it is possible to fully observe and quantify
perception and behaviour\textquotedblright{} {[}3{]}

-NEF -\textquotedblright{}The NEF provides principles to guide the
construction of a neural model that incorporates anatomical constraints,
functional objectives, and dynamical systems or control theory. Constructing
models from this starting point, rather than from single cell electrophysiology
and connectivity statistics alone, produces simulated data that explains
and predicts a wide variety of experimental results.\textquotedblright{}
{[}5{]} -Compare to data driven models and machine learning approaches
-\textquotedblright{}...NEF-designed models match physiological and
psychological fi{}ndings without being built specifi{}cally into the
design. These results are a consequence of the need to satisfy functional
objectives within anatomical and neurobiological constraints.\textquotedblright{}
{[}5{]} -NEF Connection weight simplification -NEF allows modeling
at the neural level or the cognitive science level using methods ranging
from leaky integrate and fire (LEF) spiking neurons, to dynamical
system models 

of how serial order is learned and executed in response to needs and
environmental constraints.

-Nengo description -\textquotedblright{}Nengo is a software tool that
can be used to build and simulate large-scale models based on the
NEF; currently, it is the primary resource for both teaching how the
NEF is used, and for doing research that generates specifi{}c NEF
models to explain experimental data.\textquotedblright{} {[}5{]} -\textquotedblright{}Nengo
is a graphical and scripting based software package for simulating
large-scale neural systems.\textquotedblright{} {[}2{]} -\textquotedblright{}Among
other things, Nengo has been used to implement motor control, visual
attention, serial recall, action selection, working memory, attractor
networks, inductive reasoning, path integration, and planning with
problem solving\textquotedblright{} {[}2{]} -Nengo Object model {[}5{]}
pic and descrip? -Comparison to Brian and PyNN -Updates from version
in {[}5{]}

-Paul Verschure\textquoteright{}s DAC model -DAC overview pic and
description{[}1{]} -\textquotedblright{}DAC is unique in that it has
been explored using robots and mechatronic systems (including Ada)
in a range of tasks, links between symbolic and sub/non-symbolic approaches,
has been mapped to a number of key brain systems and given rise to
novel neurorehabilitation technologies.\textquotedblright{}{[}Emb
web{]} -Opposing Models?

-Layer descriptions of DAC as they relate to our specific implementation
-Vision -Optical Flow, Pattern Rec -\textquotedblright{}DVS retina
sensing specific shapes and landmarks. Sends out events to FPGA Vertex
6 where we implement convolutional network with 9 filters. processes
events in parallel. program 9 different kernels to detect different
shapes. For certain shapes we program different sizes and angles so
that the convolutional network can detect in different angles. -Scalable
AER method using character recognition based on convolutional type
network presented in {[}{[}7{]} -AER convolution, convolutional network

-Allostatic Control -Targeted behaviors -\textquotedblright{}smooth
and robust control of stereotypical behaviors\textquotedblright{}{[}1{]}
-\textquotedblright{}one salient feature of these behaviors is that
the response intensity varies with the salience of the target stimulus\textquotedblright{}{[}1{]}
-Decision/Reward/Planning -\textquotedblright{}We found, using both
simulated and real-world robots, that performance in the static condition
{[}no behavioral feedback{]} was strongly reduced by comparison with
the enabled condition\ldots{} This result confirms that behavioural
feedback directly enhances performance.\textquotedblright{}{[}3{]}
-\textquotedblright{}Thus, we propose that the brain exploits behavioural
feedback to constrain perceptual learning and to stabilize acquired
behavioural structures.\textquotedblright{}{[}3{]} -\textquotedblright{}We
observed a signifi{}cant and systematic difference in RT and neural
response variability that held over a wide range of trial history
conditions. These results suggested that, other than perceptual signals,
neurons in PMd are also infl{}uenced by an additional input related
to the history of the trial, i.e., memory. To validate this hypothesis,
we studied the response of a mean-fi{}eld approximation of a spiking
neural model (Wilson and Cowan, 1972) in a simulated countermanding
task. We observed that an additional monitoring-related signal can
directly account for the observed changes in the neural response variability
and the behavioural performance.\textquotedblright{} {[}4{]} --Learning
Models?

-Sensory/motor -Spatial Memory

-Goals/Internal States/ value and utility Takes the allostatic outputs
and weight it to determine a stable internal state which could include
rturn/stay home (caled home, low batt power, sense danger or explore
(if search for food, has adequate batt).

-Code snippets -Nengo block diagrams


\subsubsection{Results EMB }

-Output from nengo (shots of camera feed, thresholds being crossed
etc. -How can we quantify/benchmark behavioral traits? -Since we are
integrating piece by piece, it would really be cool to have results
from each phase of integration\ldots{} That way we can show how useful
all of these layers are.


\subsubsection{Discussion EMB}

-Technologies and studies already developed due to this line of research.
-Future direction (Spinnaker integration, SPAUN +DAC=SPARTA\ldots{}) 


\subsection{Speedy attack robot}


\subsection{Prototype reactive layer foraging for pushbot}


\subsection{Egomotion compensation using gyroscope}


\subsection{Android Devices and Neuromorphic Camera}


\subsection{Extensions }

stab at one or two of those and what to add to the paper 

- cerebellum model relatively str to add 

- prototype for doing SPA model running on the robot look for 

comparisons between any robot competitions ?? what other things have
been done out there and how does our system compare 
